\documentclass[pdftex,a4paper,halfparskip]{scrartcl}
% Andere Dokumentklassen: article, report, book, beamer, scrartcl, scrreprt, scrbook, beamer
% Prefix scr bedeuted aus dem KOMA-Script Paket, welches europäische Formatierungsstandards enthält
% Gebräuchliche Optionen sind: 11pt, 12pt, twoside, twocolumn, a4paper,...
\usepackage{ngerman} %Titel z.B. für Tabellen- und Inhaltsverzeichnis werden ins Deutsche übersetzt. Ausserdem Aktivierung der korrekten Silbentrennung

\usepackage{spverbatim}

\usepackage[latin1, utf8]{inputenc} %Für das Erkennen von Umlauten

\usepackage[T1]{fontenc} %Font-Encodierung wird auf das T1-Format mit bis zu 256 (anstelle 128 im Default Fontencoder) umgeschaltet

\usepackage{hyperref}
\usepackage{graphicx}
\usepackage{color}

\title{Rapid Frontend Prototyping with Deep Learning} %Definition des Titels
\author{Simon Deussen}	%Definition des Autors
% \date{7.Nov.2010}

\begin{document}

\maketitle	

\begin{abstract}
Generierung von HTML/CSS Code anhand von pixelbasierten Screenshots und -designs. Aufbauend auf dem Pix2Code Paper \cite{Beltramelli17}, wird diese Arbeit eine eigene Implementation einer ausführlicheren DSL erstellen. Durch automatisches Erstellen von Frontend-Code können Anwendungen wie diese rapide Entwicklungszyklen realisieren. 
\end{abstract}


\tableofcontents	% Erstellung des Inhaltsverzeichnisses
%\listoffigures   % Abbildungsverzeichnis
%\listoftables    % Tabellenverzeichnis
\section{Einleitung} 

Viele Client-basierte Anwendungen brauchen ein schönes Frontend. Dieses soll gleichzeitig funktional und übersichtlich sein, sowie die Firma durch gutes Design repräsentieren. Seit Apple \& Co ist es unglaublich wichtig gutes Design im Frontend zu haben, da sich die Kunden sonst schnell eine Alternative suchen. Um nun ein großartiges Frontend zu realisieren benötigt man ein enge Kooperation von Designern und Entwicklern, da hier zwei, sich kaum überschneidende, Skillsets gebraucht werden. Im der klassischen Entwicklung sieht diese Zusammenarbeit folgendermaßen aus: \\
Ein Designer macht einen grafischen Entwurf, dieser wird vom Kunde abgenommen, dann geht er zu dem Entwickler, der nun zu aller erst Markup für den Content und anschließend das Design und die richtige Darstellung nach bauen muss. Für jede grafische Veränderung muss dieser Prozess wieder ausgeführt werden. Für die meisten Entwickler, ist die Markup und CSS-Erstellung der widrigste Part der ganzen Arbeit, da er recht zeit-aufwendig, repetitiv und langweilig ist. Es gab bisher viele Ansätze diese Arbeit zu automatisieren, zum Beispiel durch Tools in dem man gleichzeitig Designen und den Markup exportieren kann. Leider sind diese Tools entweder nicht besonders gut die Designs zu erstellen oder darin den Markup zu exportieren.

Eine Abhilfe soll diese Arbeit liefern: Sie ermöglicht, dass der Designer mit seinem bevorzugten Tools das Design baut und der Entwickler mit einem Mausklick das fertige Markup bekommt. So kann sich der Entwickler vollends auf die Realisierung des Verhaltens und der Logik der Anwendung konzentrieren. 

\section{Ähnliche Arbeiten}

Diese Arbeit basiert auf dem Pix2Code Paper von Tony Beltramelli \cite{Beltramelli17} . Er war der erste der Code anhand von visuellen Input generieren kann. 
Anderen Ansätze wie DeepCoder \cite{DeepCoder16} benötigen komplizierte DSL als Input und schränken so die Benutzbarkeit stark ein. Visuelle Versuche mit Android-GUIs von Nguyen \cite{Nguyen15} benötigt ebenfalls unpraktische von Experten erstellte Heuristiken. Pix2Code ist so das erste Paper das einen allgemeinen Input hat, und daraus momentan drei verschiedene Targetsprachen hat. Zum einen kann es HTMl/CSS Code erstellen, zum anderen aber auch Android- und iOS-Markup. Siehe Original Code auf Github \cite{Beltramelli17Github}


\section{Motivation}
Computer genierte Programme werden die Zukunft der Software Entwicklung sein und diesen Bereich auch grundsätzlich verändern. Schon jetzt im Bereich der Cloud mit Serverless Computing und Lambdas, geht es oftmals bestehende Software-Teile zu verbinden. Diesen Trend, der für die Reduktion des Schreibens, mehr in die Richtung des Konfigurieren geht, sehe in Zukunft noch viel umfassender und allgegenwärtig. Es wird so weit gehen, dass man nicht nur wie hier in dieser Arbeit das Markup generiert sondern das aus einer Skizze sofort eine fertige Website gebaut wird, und man mit ein paar Klicks das nötige Verhalten einfach hinzufügen kann. 
Wobei dieser Trend nicht nur auf das Web bezogen ist. Ich denke das sich die Webtechnologien auch in der Desktop Umgebung durchsetzen.  Da Plattform unabhängig und sehr stark optimiert. Sehr einfach zu lernen, weit verbreitet. Zum Beispiel Electron \cite{electron} ermöglicht den einfachen Einsatz durch einen eingebettet Browser.
Daher kommt die Motivation diese Arbeit zu verfassen: Automation ist unumgänglich, deswegen sollte man diese am besten selber bauen und damit endlosen Entwicklern das Leben leichter machen.

\section{Benutzte Technologien}

In dem folgenden Abschnitt werden die benutzten Technologien beschrieben. Diese Erklärungen sind recht generell und gehen zunächst nicht auf die genaue Verwendung der Technologien in dem Projekt ein, dies wird aber im Abschnitt Überblick genauer beleuchtet.

\subsection{Neuronale Netzwerke}
Neuronale Netzwerke sind einfach zu benutzende Modelle, welche nicht-lineare Abhängigkeiten mit vielen latenten Variablen stochastisch abbilden können \cite{nnWebsite}. Im einfachen Sinne, sind sie gerichtete Graphen, deren Knoten oder Nodes aus ihren Inputs Werte errechnen und die an die folgenden Nodes weitergeben. Hierbei werden zwischen 3 verschiedenen Arten von Nodes unterschieden:

\begin{description}
	\item[Input Nodes] Über diese Nodes bekommt das Netzwerke die Input Parameter.
	\item[Hidden Nodes] Nodes, welche das Netzwerke-interne Modell repräsentieren.
	\item[Output Nodes] Diese Nodes bilden die Repräsentation des Ergebnisses ab.
\end{description}

Nachdem die Node aus den Inputs einen Wert errechnet hat, geht dieser durch eine Aktivierungsfunktion. Diese Funktion stellt den Zusammenhang zwischen dem Input und dem Aktivitätslevel der Node her. Man unterscheidet zwischen folgenden Aktivitätsfunktionen

\begin{description}
	\item[Lineare Aktivitätsfunktion] Der einfachste Fall, linearer Zusammenhang zwischen Inputs und Output.
	\item[Lineare Aktivitätsfunktion mit Schwelle] Linearer Zusammenhang ab einem Schwellwert. Sehr nützlich um Rauschen herauszufiltern. Ein häufig genutzte Abhandlung davon:
	\begin{description}
		\item[ReLU] Hier werden nur der positive Werte weitergeleitet: \(f_x = x^+ = max(0,x) \)
	\end{description}
	\item[Binäre Schwellenfunktion] Nur zwei Zustände möglich: 0 oder 1 (oder auch -1 oder 1)
	\item[Sigmoide Aktivitätsfunktion] Benutzung entweder einer logistischen oder Tangens-Hyperbolicus Funktion. Diese Funktionen gehen bei sehr großen Werten gegen 1 und bei sehr negativen Werten gegen 0 (logistische Funktion) oder -1 (Tangens-Hyperbolicus Funktion). Diese Funktion bietet den Vorteil das sie das Aktivitätlevel begrenzt.
\end{description} 

Jede der Nodes hat eine bestimmte Anzahl an Verbindungen, diese hängt von der Art der Nodes und deren Zweck ab. Wichtig ist jedoch, das jede Node mit mehreren anderen Nodes verbunden ist, dies soll heißen, den Output mehrerer Nodes als Input zu bekommen und den eigenen Output als Input für die folgenden Nodes weiterzuleiten. Die Stärke der Abhängigkeit zwischen zwei Nodes wird als Gewicht ausgedrückt. Jede Verbindung in einem Neuronalen Netzwerk hat ein Gewicht, welches mit dem Output der vorangegangen Node multipliziert wird bevor es als Input weiter verwendet wird. TODO BIAS

Das Netzwerk-interne Modell wird in diesen Gewichten abgespeichert. Es repräsentiert also das Wissen, dass durch das Training entstanden ist.

Das Training eines Netzwerkes, ist das schrittweise Anpassen der Gewichte bis es ein gutes Modell des Problems gelernt hat. Die Stärke von Neuronalen Netzen liegt darin, aus großen Mengen von Daten Gesetzmäßigkeiten oder Patterns zu erkennen. Ein einfaches Beispiel ist die Objekterkennung. Wenn ein Netzwerk Alltagsgegenstände erkennen soll, lernt es die Pixelgruppen welche ein Tisch von einem Bett unterscheiden. Damit dies funktioniert braucht man eine große Menge an Daten. Zunächst wird das Training in zwei verschieden Arten unterteilt: 

\begin{description}
	\item[Supervised learning] Innerhalb des Trainingsdatensets, hat jeder Datensatz einen vorgegeben Output Label. Zum Beispiel ein Bild von einem Auto ist auch so gekennzeichnet. Nun werden so lange die Gewichte des Netzwerkes optimiert, bis ein jeweiliger Input auch den richtigen Output erzeugt.
	\item[Unupervised learning] Hier hat das Trainingsdatenset keine Label. Die Gewichtsveränderungen erfolgen im Bezug zur der Ähnlichkeit von Inputs. Das soll heißen, wenn es viele verschiedene Bilder bekommt, werden Bilder mit ähnlichen Inhalten eine hohe Nähe aufweisen, ein Bild von einem PKW wird näher an dem Bild von einem LKW sein als an dem Bild von einem Apfel.
\end{description}

\subsection{Convolutional Neural Network - CNN}

CNN sind Tiefe Neuronale Netzwerke mit einer bestimmten Architektur und spezialisiert auf die Verarbeitung von Bildern. Da man die Anwendungsdomäne eingeschränkt hat, kann man bestimmte Annahmen treffen, welche die Anzahl der Verbindungen und damit Rechenoperationen verringert und somit das Netz effektiver macht. Um aus Bildern, Informationen zu gewinnen, müssen die Ebenen des Netzwerkes nicht vollständig verbunden sein. Stattdessen werden Filter (Convolutions) und Sub-Sampling genutzt. Filter sind kleine Matrizen, die bestimmte Features entdecken, zum Beispiel Kanten mit bestimmter Ausrichtung. Durch das Erlernen der Filter im Training kann das Netzwerk aus den Pixel Werten, schrittweise abstraktere Features errechnen. Diese gehen von einfachen Kanten, zu komplexeren Umrissen, und schließlich zu vollständigen Teil-Objekten. Zum Beispiel werden aus vielen Kanten ein Kreis, dann kommen noch mehr dazu bis eine Feautere Map ein Auge abbildet. 
Aus dem Auge und der biologischen Signal Verarbeitung ist diese Architektur inspiriert \cite(Hubel68). Einzelne Neuronen des celebralen Kortex reagieren auf Reize nur in einem beschränkten Bereich. Da diese Bereiche leicht überlappen können so diese Neuronen den gesamten Sichtbereich erkennen. 

\subsection{Recurrent Neural Network - RNN}

Als RNN bezeichnet man neuronale Netze, welches Verbindungen, im Gegensatz zu FeedForward-Netzen, zu Neuronen der selben oder vorhergehenden Schichten besitzt. Dadurch kann es zeitliche Abhängigkeiten in den Input Daten detektieren. Diese Art von Netzen wird in der Spracherkennung, Maschineller Übersetzung und auch Handschrifterkennung eingesetzt. 


\subsubsection{Long short-term Memory - LSTM}

Ein LSTM ist eine bestimmte Form der RNNs. Ein RNN kann durch dessen starre Struktur immer nur eine bestimmte Anzahl von Schritten abspeichern. Zum Beispiel bei Videoanalysen jeweils die letzten 5 Frames. Das LSTM kann sich dynamisch Daten speichern, wodurch es irrelevante Daten aus vorherigen Schritten verwirft, relevantere jedoch länger abspeichert. Während des Trainings eines LSTMs, erlernt dieses auch das Speichern und Löschen. Dadurch kann es sehr viel effizienter als reine RNNs Daten mit temporaler Dimension auswerten. 


\subsection{Training}

Als Training wird der Prozess bezeichnet, während dem ein Neurales Netzwerk Wissen aus vorliegenden Daten extrahiert. Genau dieser Vorgang sorgt für den großen Erfolg von Neuralen Netzen. Anders als bei herkömmlichen Statistischen Methoden können NNs aus riesigen Datenmengen Patterns und Sachverhältnisse lernen. Damit dies funktioniert, muss eine Kostenfunktion gebildet werden können, anhand bestimmt wird, wie weit das genutzte Modell von der Optimalen Lösung entfernt ist. Anhand diesem Abstand können die Parameter des gewählten Modells angepasst werden um dem Optimum näher zu kommen.

\subsection{Sampling}
Während dem Sampling, wird ein fertig trainiertes Neurales Netzwerk genutzt um aus Daten Schlussfolgerungen abzuleiten. In dem Fall dieser Arbeit, wird probiert aus einem Screenshot, eine deutungsvolle DSL-Sequenz zu erstellen.

\subsection{Keras}
Um die in dieser Arbeit genutzten Tiefen Neuronalen Netzwerke zu definieren wir das Framework Keras genutzt. Es ist eine Library die es sehr einfach macht Neuronale Netzwerke zu erstellen und verwalten. Es selber verwaltet nur die Definition von Modellen, die eigentlichen Berechnungen werden im Backend getätigt, von Theano, Tensorflow oder CNTK. So ermöglich es schnelles experimentieren mit wenig Overhead und keinen Boilerplate-Code. 

\subsection{Domain-specific language - DSL}

Eine Programmiersprache, die auf ein einzelne Problem-Domäne spezialisiert ist, wird DSL genannt. Im Gegensatz zur DSL steht die General Purpose Language, welches ein Programmiersprache ist, die sehr breit, für viele verschiedene Anwendungen, benutzt werden kann. Die Trennung zwischen DSL und GPL ist nicht immer klar, es kann zum Beispiel Teile einer Sprache geben die hoch spezialisiert für eine bestimmte Aufgabe sind, aber andere Teile von ihr können allgemeinere Aufgaben lösen. Auch historisch bedingt kann sich die Einordnung einer Sprache ändern. JavaScript wurde ursprünglich für ganz einfache Steuerung von Websites eingeführt, kann aber inzwischen für alles mögliche eingesetzt werden - vom Trainieren von CNNs im Browser, zu klassischen Backend-Jobs. 
In dieser Arbeit, wird eine hochspezialisierte, eigens erstellte Sprache aus Token, die eine Kombination aus HTML und CSS sind, benutzt.

\subsubsection{Hypertext Markup Language - HTML}
HTML ist die Standard Programmiersprache um Websites zu erstellen. Mit einzelnen HTML Elementen beschreibt es den semantischen Zusammenhang des Contents von Websites.

\subsubsection{Cascading Style Sheets - CSS}
CSS beschreibt die Präsentation, also das Aussehen, des Content einer Markup-Language (zum Beispiel HTML). Klassische Inhalte, sind Farben, Positionen und Effekte von User Interface Elementen.
 

\section{Rapid Frontend Prototyping - Überblick}

RFP ist ein Tool. welches aus einfachen GUI-Screenshots HTML/CSS Markup erzeugen kann. Dafür lernt ein System aus Neuralen Netzwerken eine einfache Token Sprache / DSL. Jedes Wort dieser Sprache hat zugehörigen HTML/CSS Markup, wodurch es nach Erstellung der Tokens einfach in diese Sprache kompiliert werden kann.

Um den Code für die Website Screenshots zu generieren, wird ein Modell bestehend aus 2 LSTMs und einem CNN erstellt. Das hier benutzte Modell ist eine Abwandlung des im Original Paper verwendeten Modells. Da die neue DSL komplexer als die im Original Paper ist, musste sowohl das CNN als auch die LSTMs verändert werden. 

Von einem Computer abgespeicherte Bilder sind ein denkbar schlechtes Format um aus den Rohdaten auf Inhalte zuschließen. Zu Grunde liegen ist jedes Bild ein oder mehrere Matrizen deren einzelne Elemente einem Pixel zugeordnet werden. Um aus diesem Zahlenwerten Schlussfolgerungen zu schließen, müssen diese in mehreren Schritten abstrahiert werden um zu einem bedeutungsvollem Datenmodell zu kommen. Hierfür wird ein CNN genutzt. Dieses verarbeitet das gegebene Input-Bild und erstellt eine niedrig-dimensionalere Repräsentation. Die Architektur des CNN ist recht simple und angelehnt an VGGNet von Simonyan and Zisserman. 

Das Training des Systems erfolgt über ein Sliding-Window-Prinzip, jede Token Sequenz wird mit einem Fenster einer bestimmten Context-länge $n$ aufgeteilt, und jeder einzelne Token wird mit seinen jeweiligen $n$ Nachbarn sowie dem Trainingsbild als Input genutzt. Dadurch entsteht eine gewaltige Anzahl an Trainingsdaten, da jedes Trainingsbild $t$ mal genutzt wird. $t$ ist hierbei die Anzahl der Token für das jeweilige Bild. 

Bei einer Kontext Länge von 96, gibt es insgesamt 377.457 Bild Token-Sequenz Paare. CHECK DAS AUS OB ES ÜBERAHUTP VON DER LÄNGE ABHÄNGT. Diese werden mit Mini-Batches a 64 Bild-Token-Sequenz Paare abgearbeitet. 


\section{Daten Synthese}

Da im Zuge dieser Arbeit eine Erweiterung der DSL des Original Papers implementiert wurde, ist es erforderlich, neue Trainingsdaten zu synthetisieren. Das DataCreationTool ist ein Programm, welches nach vorgegebenen Regeln einen Token-Baum erzeugt und diesen anschließend abspeichert. Dieser Token-Baum hat immer einen body-Token als Wurzel und der gesamten Inhalt liegt als dessen Kinder vor. Dafür wurde eine Helfer Klasse geschrieben, die ein Element in dem Token Baum abbildet. Diese kann zum einen Parameter wie den Token-Name, Inhalt und Kinder speichern, zum anderen enthält sie Funktionen zum Konvertieren des Baumes zu einer String-Repräsentation sowie zum Rendering nach HTML/CSS. 

\subsection{Generieren der Token-Bäume}
Die Token-Bäume werden in der Datei \texttt{createAllTokens.py} generiert. Diese Datei erzeugt alle möglichen Token-Kombination anhand der folgenden Regeln:

\subsubsection{Gramatik}

\begin{equation}
start \rightarrow [H,C]
\end{equation}

\begin{equation}
H \rightarrow [Ml | Mr | S]
\end{equation}
\begin{equation}
Ml \rightarrow  [ logoLeft, buttonWhite | logoLeft, buttonWhite, buttonWhite | ...]
\end{equation}
\begin{equation}
Mr \rightarrow [buttonWhite, logoRight | buttonWhite, buttonWhite, logoRight | ...]
\end{equation}
\begin{equation}
S \rightarrow [sidebarHeader, sidebarItem| sidebarHeader, sidebarItem, sidebarItem | ...]
\end{equation}

\begin{equation}
C \rightarrow [R | R, R | R, R, R ]
\end{equation}
\begin{equation}
R \rightarrow [S | D, D, | Q, Q, D | Q, D, Q | D, Q, Q | Q, Q, Q, Q]
\end{equation}
\begin{equation}
S, D, Q \rightarrow [smallTitle, text, contentButton]
\end{equation}
\begin{equation}
contentButton \rightarrow [buttonBlue, buttonGrey, buttonBlack]
\end{equation}

Regeln 3 - 5 sind gekürzt. Es können bis zu 5 Buttons auftreten.

\subsubsection{Zeichenerklärung}

\begin{description}
	\item[H] Header der Website, enthält eins der folgenden Elemente: 
	\begin{description}
		\item[Ml] Menue mit Logo auf der linken Seite
		\item[Mr] Menue mit Logo auf der rechten Seite
		\item[S] Sidebar	
	\end{description}
	\item[C] Content der Website, besteht aus ein bis drei Wiederholungen dieses Elements:
	\begin{description}
		\item[R] Row, die aus einem oder mehreren Row Elementen bestehen kann:
		\begin{description}
			\item[S] Single Row Element, die ganze Row ist mit diesem ausgefüllt.
			\item[D] Double Row Element, ist so breit wie eine Hälfte der Row
			\item[Q] Quadruble Row Element, ist so breit wie ein Viertel der Row
		\end{description}
		Jedes dieser Elemente enthält den gleichen Inhalt:
		\item[smallTitle] Überschrift
		\item[text] Text-Inhalt
		\item[contentButton] Ein Button der entweder Blau, Grau oder Schwarz ist 
	\end{description}
\end{description}

\subsubsection{Token Generierung}

Die Generierung erfolgt mit einfachen, verschachtelten Loops und einem Kartesischen Produkt, um alle möglichen Kombinationen abdecken. Dadurch werden 4128 verschiedene Layout Kombinationen möglich. 
Die Layout Kombinationen haben eine durchschnittliche Länge von 62.47 Token (Arithmetisches Mittel), bei einen Median von 65 Token. Außerdem ist die Maximale Länge 92, die Minimale Token Anzahl ist 16.

\begin{verbatim}
    menu_or_sidebar = [True, False]
    logo_left_or_right = [True, False]
    possible_num_of_menu_button = [1, 2, 3, 4]
    possible_num_of_rows = [1,2,3]
    possible_row_type = [0,1,2,3,4]

    row_count_layout_combinations = []

    for i in possible_num_of_rows:
        row_count_layout_combinations.extend( list(itertools.product(possible_row_type, repeat=i)))

    for i in range(len(row_count_layout_combinations)):
        row_count_layout_combinations[i] = list(row_count_layout_combinations[i])

    complete_layouts =  []

    for menu_flag in menu_or_sidebar:
        for logo_flag in logo_left_or_right:
            for num_of_menue_button in possible_num_of_menu_button:
                for row_count_layout in row_count_layout_combinations:

                        root = Element("body", "")

                        if menu_flag:
                            menu = tokenBuilder.createMenu(logo_flag, num_of_menue_button)
                            root.addChildren(menu)
                        else:
                            sidebar = tokenBuilder.createSidebar(num_of_menue_button)
                            root.addChildren(sidebar)

                        for i in range(len(row_count_layout)):
                            row = tokenBuilder.createRow(row_count_layout[i])
                            root.addChildren(row)

                        complete_layouts.append(root)
\end{verbatim}

In den ersten fünf Zeilen werden die jeweiligen Konfigurations Möglichkeiten in Listen geschrieben. Anschließend wird eine weitere Liste erstellt mit allen Möglichen Kombinationen aus Anzahl von Rows und Row Type mit der Funktion itertool.product(). Um dies zu erreichen hätte man auch je nach Anzahl an Rows verschachtelte For-Loops benutzten können, dieser Ansatz wäre jedoch zu unflexibel falls in Zukunft noch mehr Rows hinzukommen würden.
Als letzter Schritt wird eine Loop pro Liste benutzt um über den gesamten Raum der möglichen Kombinationen zu Iterieren. Dann wird mit der jeweiligen Kombination ein Token-Baum gebildet und der Liste aller Token-Bäume angehängt. Im weiteren Code verlauf wird diese Liste auf mehrerer Threads verteilt und parallel in eigenen Files gespeichert.
Bei der Speicherung wird wie im Fall des Renderings eine rekursive Funktion auf Element-ebene ausgeführt, die den Tag jedes Elements und der Kinder in einen String zusammenführt.

\subsection{DSL Mapping}

Zu jedem dieser Token, existiert ein Mapping nach HTML/CSS. In einer extra Datei, \texttt{dsl-mapping.json} ist dies abgebildet:

\begin{verbatim}
  {
    "opening-tag": "{",
    "closing-tag": "}",
    "body": "<!DOCTYPE html>\n <head>\n <meta charset=\"utf-8\">\n ...
    "header": "<nav class=\"menue\">\n    <ul class=\"nav nav-pills...
    "btn-active": "<li class=\"active\"><a href=\"#\">[]</a></li>\n...
    "btn-inactive-blue": "<button type=\"button\" class=\"btn btn-p...
    "btn-inactive-black": "<button type=\"button\" class=\"btn btn-...
    "btn-inactive-white": "<button type=\"button\" class=\"btn btn-...
    "btn-inactive-grey": "<button type=\"button\" class=\"btn btn-p...
    "row": "<div class=\"container\"><div class=\"row\">{}</div></d...
    "single": "<div class=\"col-lg-12\">\n{}\n</div>\n"
    "double": "<div class=\"col-lg-6\">\n{}\n</div>\n",
    "quadruple": "<div class=\"col-lg-3\">\n{}\n</div>\n",
    "big-title": "<h2>[]</h2>",
    "small-title": "<h4>[]</h4>",
    "text": "<p>[]</p>\n",
    "logo-left": "<a class=\"logo-left\">RFP</a>\n",
    "logo-right": "<a class=\"logo-right\">RFP</a>\n",
    "sidebar": "<div class=\"wrapper\">\n    <div id=\"sidebar\">\...
    "sidebar-element": "<li><a href=\"#\">[]</a><li>"  
  }
  
\end{verbatim}

Um somit aus einem Token-Baum HTML/CSS Markup zu erzeugen, startet der Wurzelknoten eine rekursive Rendering-Funktion. Diese nutzt den zu den Knoten gehörigen Code und traversiert den gesamten Baum. Jeder Mapping String eines Tokens, der auch Kinderknoten haben kann, enthält einen Platzhalter, hier \texttt{\{\}}, mit dem signalisiert wird, wo der Code der Kinderknoten hingehört. Ähnlich gibt es ebenso einen Platzhalter für Text-Content, nämlich die Zeichen: \texttt{[]}. 
Für den Text-Content wird im Zuger der Arbeit nur zufällige Zeichenfolgen genommen, damit das Neurale Netzwerk lernt nicht auf den Text zu achten. 


\subsection{Screenshot Erstellung}

Nachdem der HTML/CSS String als Datei abgespeichert wurde, kann mit dem Tool \texttt{imgkit} ein \texttt{.png} oder \texttt{.jpg} erzeugt werden.

\subsection{Teilen der Trainingsdaten}

Die Gesamten Trainingsdaten, werden in ein Trainingsset mit einem Anteil von 70\%, einem Testset mit 20\% und einem Validierungsset mit 10\% der Bilder abgespeichert

\section{Validierung der Ergebnisse}
Um die Qualität der Ergebnisse zu vergleichen, stelle ich eine Metrik auf, die Anhand der Wichtigkeit der jeweiligen Elemente einen Score erstellt. Auf Token-Ebene einfach nach Fehlern zu suchen und jeden Falschen Token gleich zu gewichten, würde zu nicht aussage-kräftigen Scores leiten. Zum Beispiel ist ein Menue-Item zu viel oder zu wenig, sehr viel weniger schlimm, als eine falsche Kategorisierung des Headers der Website. Um eine ständig gleich bleibende Bewertung der trainierten Netzwerke zu schaffen, ist eine Datei \texttt{evaluateModel.py} implementiert worden. Diese wird nach jedem Trainieren einen Score mit allen Testdaten errechnen. Dazu werden pro Datensatz mehrere Tests gemacht. Der erste Test überprüft ob der Header korrekt ist. Dieser ist einer der am stärksten gewichteten. Danach wird überprüft ob die Anzahl der Menue Item korrekt. Anschließend wird der Rest des Content überprüft, und zwar die Anzahl an Rows, der richtige Row Type pro Row und die Anzahl der korrekten sowie falschen Buttons. Pro Datensatz wird so ein \texttt{error\_object} erstellt. 
Hier ein Beispiel:

\begin{spverbatim}

{
 'countCorrectButtons': 1,
 'countCorrectRowType': 0,
 'countWrongButtons': 5,
 'countWrongRowType': 3,
 'differenceButtonCount': 3,
 'differenceMenuButtons': 0,
 'differenceRowCount': 0,
 'differenceTokenCount': 16,
 'isHeaderCorrect': True,
 'predictedFileName': './generatedMarkup/second.gui',
 'trueFileName': './generatedMarkup/SECOND_complete_generation_4_04.10.2018_1538655331945.gui',
 'trueHeaderType': 'sidebar',
 'true_token_count': 55
}

\end{spverbatim}

Hier hat das Modell insgesamt 16 Tokens zu viel generiert, diese kommen aus zu drei zu viel generierten Buttons, sowie den falschen Row Types. Jede generierte Row hat hier den Falschen Row Type, zu sehen an \texttt{'countCorrectRowType': 0}. Richtig generiert wurde der Header Type, hier wurde die korrekte \texttt{sidebar} verwendet.

\section{Vergleich zu dem Original}
\section{Experimente}

\subsubsection{1. Trainingsversuch}

Um eine Baseline aller weiteren Ergebnisse zu erstellen, wird das Original Modell aus dem pix2code Paper mit meinen neuen Daten neu trainiert. Dazu wurden alle Einstellungen gelassen wie sie waren. Da in dem benutzten Datenset eine andere Anzahl von Token ist, musste dies in 2 Layern angepasst werden:

\begin{verbatim}

_____________________________________________________________________________________
Layer (type)                    Output Shape         Param #     Connected to

=====================================================================================
input_1 (InputLayer)            (None, 256, 256, 3)  0
_____________________________________________________________________________________
input_2 (InputLayer)            (None, 48, 24)       0
_____________________________________________________________________________________
sequential_1 (Sequential)       (None, 48, 1024)     104098080   input_1[0][0]
_____________________________________________________________________________________
sequential_2 (Sequential)       (None, 48, 128)      209920      input_2[0][0]
_____________________________________________________________________________________
concatenate_1 (Concatenate)     (None, 48, 1152)     0           sequential_1[1][0]
                                                                 sequential_2[1][0]
_____________________________________________________________________________________
lstm_3 (LSTM)                   (None, 48, 512)      3409920     concatenate_1[0][0]
_____________________________________________________________________________________
lstm_4 (LSTM)                   (None, 512)          2099200     lstm_3[0][0]
_____________________________________________________________________________________
dense_3 (Dense)                 (None, 24)           12312       lstm_4[0][0]
=====================================================================================
Total params: 109,829,432
Trainable params: 109,829,432
Non-trainable params: 0
_____________________________________________________________________________________

CONTEXT_LENGTH 48
IMAGE_SIZE 256
BATCH_SIZE 64
EPOCHS 5
STEPS_PER_EPOCH 72000

input_shape (256, 256, 3)
output_size 24
input_images 377457

\end{verbatim}

Diese Anpassung ist in in der Layer \texttt{input\_2} und \texttt{dense\_3} sichtbar. \texttt{input\_2} ist das Language Modell und \texttt{dense\_3} der Finale Output.

Leider resultierte das Training in keinen guten Ergebniss. Egal welcher Input dem Modell gegeben wurde, es erzeugt immer den gleichen Output. Nämlich diese Abfolge von Token:

\begin{verbatim}
body {
	sidebar {
		sidebar-element,sidebar-element,sidebar-element,sidebar-element
	},
	row {
		quadruple {
			small-title,text,btn-inactive-blue
		},
		quadruple {
			small-title,text,btn-inactive-blue
		},
		double {
			small-title,text,btn-inactive-blue
		}
	},
	row {
		quadruple {
			small-title,text,btn-inactive-blue
		},
		quadruple {
			small-title,text,btn-inactive-blue
		},
		double {
			small-title,text,btn-inactive-blue
		}
	},
	row {
		quadruple {
			small-title,text,btn-inactive-blue
		},
		quadruple {
			small-title,text,btn-inactive-blue
		},
		double {
			small-title,text,btn-inactive-blue
		}
}
\end{verbatim}

Bei jedem Input, kommt mit ungefähr gleich hohen Wahrscheinlichkeiten immer diese Sequenz heraus. Deswegen wird angenommen, dass diese Abfolge von Token die allgemeingültigste ist. Bei diesem Output kommt der geringste Fehler heraus, obwohl sie nicht mal ein gültiges Dokument darstellt.

\subsection{2. Trainingsversuch}

In diesem Versuch wurde eine vergrößerte Context-Länge erprobt. Da im ersten Versuch eine das Training nicht funktioniert hat, vermute ich den Fehler in einem zu kleinen Sichtbereich des LSTMs. Da durch die Token-Länge die Sichtbarkeit der Long-Term-Dependencies geregelt wird, verdoppel ich diese um zu sehen ob dadurch ein verbessertes Ergebnis zustande kommt. 

Das Modell ist genau gleich wie im ersten Versuch, lediglich die Context-Länge wurde von 48 auf 96 erhöht.

Wie bei ersten Versuch ergibt dieses Training wieder der gleichen Output. Egal welches Bild als Input genommen wird. Um sicher zu gehen, dass diese Prediction nicht zufällig von dem gewählten Testbild abhängt, habe ich insgesamt 313 Bilder getestet, um festzustellen das alle diese Bilder zum gleichen Ergebnis führen. Alle dieser Bilder bekommen den exakt gleichen Output

\subsection{3. Trainingsversuch}

Da im 1. und 2. Versuch, mit der Original Architektur kein brauchbares Ergebnis heraus kommt, und das Netz ein zu allgemeines Ergebnis liefert, wird nun versucht mit einer höheren Komplexität der LSTMs ein Modell zu finden das besser funktioniert.

Hier hab ich die Anzahl der LSTM \texttt{units} von 128, im Language Modell, auf 192 erhöht. Außerdem erfolgte eine Veränderung der LSTM \texttt{units} von 512 auf 768 im Decoder Modell.

\begin{spverbatim}

_____________________________________________________________________________________
Layer (type)                    Output Shape         Param #     Connected to
=====================================================================================
input_1 (InputLayer)            (None, 256, 256, 3)  0
_____________________________________________________________________________________
input_2 (InputLayer)            (None, 48, 24)       0
_____________________________________________________________________________________
sequential_3 (Sequential)       (None, 48, 1024)     104098080   input_1[0][0]
_____________________________________________________________________________________
sequential_4 (Sequential)       (None, 48, 192)      462336      input_2[0][0]
_____________________________________________________________________________________
concatenate_1 (Concatenate)     (None, 48, 1216)     0           sequential_3[1][0]
                                                                 sequential_4[1][0]
_____________________________________________________________________________________
lstm_3 (LSTM)                   (None, 48, 768)      6097920     concatenate_1[0][0]
_____________________________________________________________________________________
lstm_4 (LSTM)                   (None, 768)          4721664     lstm_3[0][0]
_____________________________________________________________________________________
dense_3 (Dense)                 (None, 24)           18456       lstm_4[0][0]
=====================================================================================
Total params: 115,398,456
Trainable params: 115,398,456
Non-trainable params: 0

_____________________________________________________________________________________

CONTEXT_LENGTH 48
IMAGE_SIZE 256
BATCH_SIZE 64
EPOCHS 5
STEPS_PER_EPOCH 72000

input_shape (256, 256, 3)
output_size 24
input_images 377457

\end{spverbatim}

Da inzwischen immer noch der gleiche Output generiert wird, probiere ich nun einen Fehler im Programm-Code zu finden. Folgene Ursachen könnte es geben:

\begin{description}
	\item[Fehler im Sampler] Der Sampler generiert nach dem Training aus Screenshots die Token-Sequenz. Hier könnte zum einen, ein falsches Modell geladen werden, oder es wird fehlerhaft geladen.
	
	\begin{description}
		\item[Falsches Modell] Durch Umbenennung des Modells im Output Folder und Eingabe eines falschen Modellnamens, bricht das Programm jeweils ab.
		\item[Fehlerhaftes Laden] Die Ausgabe von \texttt{model.summary()} nach dem Laden des Modells ist identisch mit der erstellten während des Trainings.
	\end{description}
	
	\item[Fehler in den Daten] Während dem Preprocessing werden die Trainingsbilder in \texttt{numpy}-Arrays mit \texttt{shape(256,256,3} umgewandelt. Anschließend werden die Arrays und \texttt{gui}-Files zusammen abgespeichert.
	
	\begin{description}
		\item[Array Konvertierung] Beweis Bild + Nach optischer Kontrolle von zufälligen $n=10$ Samples musste ich feststellen, dass diese korrekt sind.
		\item[Zuordnung Files] Nach kontrolle zufälliger Samples sieht die Zuordnung ebenfalls gut aus.
	\end{description}
	
	\item[Fehler in Vocabulary] Es könnte auch an der fehlerhaften Abspeicherung der einzelnen Tokens liegen. Vocabulary und One-Hot-Encoding sind ebenfalls richtig.
	\item[Fehler im Datenset / -synthese] Hier wurde ein Fehler gefunden: Alle Screenshots, welche die Sidebar anstatt des Menues haben, sind doppelt enthalten. Der Grund hierfür, war die Positionierung des Menü-Logos, welches Einfluss auf die Erstellung des Menüs hatte, aber nicht auf die Sidebar. Ein Viertel der Trainingsdaten hat ein Menü mit Logo links, ein weiteres Viertel hat das Logo rechts. Die andere hälfte der Trainingsdaten hat die Sidebar, obwohl sich dort nichts mehr ändert. 
\end{description}

\subsection{4. Trainingsversuch}

Nach dem finden des Fehlers nach dem dritten Versuch in den synthetisierten Trainingsdaten, wurde dieser behoben und neue Trainingsdaten generiert. Nun sind es nur noch 3096 verschiedene Bilder. Ein Drittel hat das Feature \texttt{menu\_logo\_left}, ein Drittel \texttt{menu\_logo\_right}, und der Rest die \texttt{sidebar}. In diesem Versuch werden wieder die Standard Parameter benutzt. 


Es erfolgte keine Verbesserung der Trainingsergebnisse. Die Token-Sequenz hat sich dahingehend geändert, dass nun anstatt der \texttt{sidebar} ein Menü in allen Ergebnissen ist.

\begin{spverbatim}
	body{
	header{
	logo-right,btn-inactive-white,btn-inactive-white
	}
row{
quadruple{
small-titletextbtn-inactive-black
}
quadruple{
small-titletextbtn-inactive-black
}
double{
small-titletextbtn-inactive-black
}
}
row{
quadruple{
small-titletextbtn-inactive-black
}
quadruple{
small-titletextbtn-inactive-black
}
double{
small-titletextbtn-inactive-black
}
}
}
\end{spverbatim}


bin7
Nach dem Training wird das traininerte netz weiter trainiert 

\begin{verbatim}
Generating sparse vectors...Dataset size: 283613Vocabulary size: 24Input shape: (256, 256, 3)Output size: 24Parsing data...WARNING:tensorflow:From /home/sd092/.direnv/python-3.6.6/lib/python3.6/site-packages/keras/backend/tensorflow_backend.py:1264: calling reduce_prod (from tensorflow.python.ops.math_ops) with keep_dims is deprecated and will be removed in a future version.Instructions for updating:keep_dims is deprecated, use keepdims insteadWARNING:tensorflow:From /home/sd092/.direnv/python-3.6.6/lib/python3.6/site-packages/keras/backend/tensorflow_backend.py:1247: calling reduce_sum (from tensorflow.python.ops.math_ops) with keep_dims isdeprecated and will be removed in a future version.Instructions for updating:keep_dims is deprecated, use keepdims insteadWARNING:tensorflow:From /home/sd092/.direnv/python-3.6.6/lib/python3.6/site-packages/keras/backend/tensorflow_backend.py:1349: calling reduce_mean (from tensorflow.python.ops.math_ops) with keep_dims is deprecated and will be removed in a future version.Instructions for updating:keep_dims is deprecated, use keepdims instead2018-10-11 15:11:57.143470: I tensorflow/core/common_runtime/gpu/gpu_device.cc:1484] Adding visible gpu devices: 02018-10-11 15:11:57.143516: I tensorflow/core/common_runtime/gpu/gpu_device.cc:965] Device interconnect StreamExecutor with strength 1 edge matrix:2018-10-11 15:11:57.143521: I tensorflow/core/common_runtime/gpu/gpu_device.cc:971]      02018-10-11 15:11:57.143525: I tensorflow/core/common_runtime/gpu/gpu_device.cc:984] 0:   N2018-10-11 15:11:57.143764: I tensorflow/core/common_runtime/gpu/gpu_device.cc:1097] Created TensorFlow device (/job:localhost/replica:0/task:0/device:GPU:0 with 11362 MB memory) -> physical GPU (device: 0, name: TITAN Xp, pci bus id: 0000:05:00.0, compute capability: 6.1)Epoch 1/104432/4431 [==============================] - 1848s 417ms/step - loss: 0.1798Epoch 2/104432/4431 [==============================] - 1843s 416ms/step - loss: 0.1798Epoch 3/104432/4431 [==============================] - 1844s 416ms/step - loss: 0.1801Epoch 4/104432/4431 [==============================] - 1845s 416ms/step - loss: 0.1796Epoch 5/10 647/4431 [===>..........................] - ETA: 26:27 - loss: 0.1814^[[A^[[A^[[A^[[A^[[A^[[A^[[A^[[4432/4431 [==============================] - 1883s 425ms/step - loss: 0.1810[B^[[B^[[B^[[B^[[B^[[B^[[Epoch 6/10[[B^[[B^[[B^[[B^[[B^[[B^[[B^[[B^[[B^[[B^[[B^[[B4432/4431 [==============================] - 1883s 425ms/step - loss: 0.1809Epoch 7/104432/4431 [==============================] - 1959s 442ms/step - loss: 0.1798Epoch 8/104432/4431 [==============================] - 1858s 419ms/step - loss: 0.1801Epoch 9/104432/4431 [==============================] - 1857s 419ms/step - loss: 0.1802Epoch 10/104432/4431 [==============================] - 1856s 419ms/step - loss: 0.1795
\end{verbatim}

\subsection{5. Trainingsversuch}

Gleichzeitig zu dem 4. Versuch wurde noch ein Subset der Daten erstellt, das nur 70\% der Bilder aufweist. Dieses hat 2166 Bilder im Vergleich zu dem Set vom 4. Trainingsversuch. Der Gedanke hierbei ist, dass im Original Training vom pix2code Paper 1500 Trainingsbilder ausgereicht haben.

Es erfolgte keine Verbesserung der Trainingsergebnisse.
bin8


\subsection{6. Trainingsversuch}

Update Training mit Löschung des Body Tags
bin9

\subsection{7. Trainingsversuch}

Versuch mit GRUs \cite{GRUvsLSTM}
bin10

Using TensorFlow backend.
Loading data...
Generating sparse vectors...
Dataset size: 256631
Vocabulary size: 23
Input shape: (256, 256, 3)
Output size: 23
\subsection{8. Trainingsversuch}

Teste irgendwie den image input... entweder traininire nen autoencoder oder validieren die ergebnisse irgendwie anders




\section{Zusammenfassung}
\section{Fazit}

\bibliographystyle{plain}
\bibliography{sources} 

\end{document}


